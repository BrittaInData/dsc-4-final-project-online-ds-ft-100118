
% Default to the notebook output style

    


% Inherit from the specified cell style.




    
\documentclass[11pt]{article}

    
    
    \usepackage[T1]{fontenc}
    % Nicer default font (+ math font) than Computer Modern for most use cases
    \usepackage{mathpazo}

    % Basic figure setup, for now with no caption control since it's done
    % automatically by Pandoc (which extracts ![](path) syntax from Markdown).
    \usepackage{graphicx}
    % We will generate all images so they have a width \maxwidth. This means
    % that they will get their normal width if they fit onto the page, but
    % are scaled down if they would overflow the margins.
    \makeatletter
    \def\maxwidth{\ifdim\Gin@nat@width>\linewidth\linewidth
    \else\Gin@nat@width\fi}
    \makeatother
    \let\Oldincludegraphics\includegraphics
    % Set max figure width to be 80% of text width, for now hardcoded.
    \renewcommand{\includegraphics}[1]{\Oldincludegraphics[width=.8\maxwidth]{#1}}
    % Ensure that by default, figures have no caption (until we provide a
    % proper Figure object with a Caption API and a way to capture that
    % in the conversion process - todo).
    \usepackage{caption}
    \DeclareCaptionLabelFormat{nolabel}{}
    \captionsetup{labelformat=nolabel}

    \usepackage{adjustbox} % Used to constrain images to a maximum size 
    \usepackage{xcolor} % Allow colors to be defined
    \usepackage{enumerate} % Needed for markdown enumerations to work
    \usepackage{geometry} % Used to adjust the document margins
    \usepackage{amsmath} % Equations
    \usepackage{amssymb} % Equations
    \usepackage{textcomp} % defines textquotesingle
    % Hack from http://tex.stackexchange.com/a/47451/13684:
    \AtBeginDocument{%
        \def\PYZsq{\textquotesingle}% Upright quotes in Pygmentized code
    }
    \usepackage{upquote} % Upright quotes for verbatim code
    \usepackage{eurosym} % defines \euro
    \usepackage[mathletters]{ucs} % Extended unicode (utf-8) support
    \usepackage[utf8x]{inputenc} % Allow utf-8 characters in the tex document
    \usepackage{fancyvrb} % verbatim replacement that allows latex
    \usepackage{grffile} % extends the file name processing of package graphics 
                         % to support a larger range 
    % The hyperref package gives us a pdf with properly built
    % internal navigation ('pdf bookmarks' for the table of contents,
    % internal cross-reference links, web links for URLs, etc.)
    \usepackage{hyperref}
    \usepackage{longtable} % longtable support required by pandoc >1.10
    \usepackage{booktabs}  % table support for pandoc > 1.12.2
    \usepackage[inline]{enumitem} % IRkernel/repr support (it uses the enumerate* environment)
    \usepackage[normalem]{ulem} % ulem is needed to support strikethroughs (\sout)
                                % normalem makes italics be italics, not underlines
    

    
    
    % Colors for the hyperref package
    \definecolor{urlcolor}{rgb}{0,.145,.698}
    \definecolor{linkcolor}{rgb}{.71,0.21,0.01}
    \definecolor{citecolor}{rgb}{.12,.54,.11}

    % ANSI colors
    \definecolor{ansi-black}{HTML}{3E424D}
    \definecolor{ansi-black-intense}{HTML}{282C36}
    \definecolor{ansi-red}{HTML}{E75C58}
    \definecolor{ansi-red-intense}{HTML}{B22B31}
    \definecolor{ansi-green}{HTML}{00A250}
    \definecolor{ansi-green-intense}{HTML}{007427}
    \definecolor{ansi-yellow}{HTML}{DDB62B}
    \definecolor{ansi-yellow-intense}{HTML}{B27D12}
    \definecolor{ansi-blue}{HTML}{208FFB}
    \definecolor{ansi-blue-intense}{HTML}{0065CA}
    \definecolor{ansi-magenta}{HTML}{D160C4}
    \definecolor{ansi-magenta-intense}{HTML}{A03196}
    \definecolor{ansi-cyan}{HTML}{60C6C8}
    \definecolor{ansi-cyan-intense}{HTML}{258F8F}
    \definecolor{ansi-white}{HTML}{C5C1B4}
    \definecolor{ansi-white-intense}{HTML}{A1A6B2}

    % commands and environments needed by pandoc snippets
    % extracted from the output of `pandoc -s`
    \providecommand{\tightlist}{%
      \setlength{\itemsep}{0pt}\setlength{\parskip}{0pt}}
    \DefineVerbatimEnvironment{Highlighting}{Verbatim}{commandchars=\\\{\}}
    % Add ',fontsize=\small' for more characters per line
    \newenvironment{Shaded}{}{}
    \newcommand{\KeywordTok}[1]{\textcolor[rgb]{0.00,0.44,0.13}{\textbf{{#1}}}}
    \newcommand{\DataTypeTok}[1]{\textcolor[rgb]{0.56,0.13,0.00}{{#1}}}
    \newcommand{\DecValTok}[1]{\textcolor[rgb]{0.25,0.63,0.44}{{#1}}}
    \newcommand{\BaseNTok}[1]{\textcolor[rgb]{0.25,0.63,0.44}{{#1}}}
    \newcommand{\FloatTok}[1]{\textcolor[rgb]{0.25,0.63,0.44}{{#1}}}
    \newcommand{\CharTok}[1]{\textcolor[rgb]{0.25,0.44,0.63}{{#1}}}
    \newcommand{\StringTok}[1]{\textcolor[rgb]{0.25,0.44,0.63}{{#1}}}
    \newcommand{\CommentTok}[1]{\textcolor[rgb]{0.38,0.63,0.69}{\textit{{#1}}}}
    \newcommand{\OtherTok}[1]{\textcolor[rgb]{0.00,0.44,0.13}{{#1}}}
    \newcommand{\AlertTok}[1]{\textcolor[rgb]{1.00,0.00,0.00}{\textbf{{#1}}}}
    \newcommand{\FunctionTok}[1]{\textcolor[rgb]{0.02,0.16,0.49}{{#1}}}
    \newcommand{\RegionMarkerTok}[1]{{#1}}
    \newcommand{\ErrorTok}[1]{\textcolor[rgb]{1.00,0.00,0.00}{\textbf{{#1}}}}
    \newcommand{\NormalTok}[1]{{#1}}
    
    % Additional commands for more recent versions of Pandoc
    \newcommand{\ConstantTok}[1]{\textcolor[rgb]{0.53,0.00,0.00}{{#1}}}
    \newcommand{\SpecialCharTok}[1]{\textcolor[rgb]{0.25,0.44,0.63}{{#1}}}
    \newcommand{\VerbatimStringTok}[1]{\textcolor[rgb]{0.25,0.44,0.63}{{#1}}}
    \newcommand{\SpecialStringTok}[1]{\textcolor[rgb]{0.73,0.40,0.53}{{#1}}}
    \newcommand{\ImportTok}[1]{{#1}}
    \newcommand{\DocumentationTok}[1]{\textcolor[rgb]{0.73,0.13,0.13}{\textit{{#1}}}}
    \newcommand{\AnnotationTok}[1]{\textcolor[rgb]{0.38,0.63,0.69}{\textbf{\textit{{#1}}}}}
    \newcommand{\CommentVarTok}[1]{\textcolor[rgb]{0.38,0.63,0.69}{\textbf{\textit{{#1}}}}}
    \newcommand{\VariableTok}[1]{\textcolor[rgb]{0.10,0.09,0.49}{{#1}}}
    \newcommand{\ControlFlowTok}[1]{\textcolor[rgb]{0.00,0.44,0.13}{\textbf{{#1}}}}
    \newcommand{\OperatorTok}[1]{\textcolor[rgb]{0.40,0.40,0.40}{{#1}}}
    \newcommand{\BuiltInTok}[1]{{#1}}
    \newcommand{\ExtensionTok}[1]{{#1}}
    \newcommand{\PreprocessorTok}[1]{\textcolor[rgb]{0.74,0.48,0.00}{{#1}}}
    \newcommand{\AttributeTok}[1]{\textcolor[rgb]{0.49,0.56,0.16}{{#1}}}
    \newcommand{\InformationTok}[1]{\textcolor[rgb]{0.38,0.63,0.69}{\textbf{\textit{{#1}}}}}
    \newcommand{\WarningTok}[1]{\textcolor[rgb]{0.38,0.63,0.69}{\textbf{\textit{{#1}}}}}
    
    
    % Define a nice break command that doesn't care if a line doesn't already
    % exist.
    \def\br{\hspace*{\fill} \\* }
    % Math Jax compatability definitions
    \def\gt{>}
    \def\lt{<}
    % Document parameters
    \title{index}
    
    
    

    % Pygments definitions
    
\makeatletter
\def\PY@reset{\let\PY@it=\relax \let\PY@bf=\relax%
    \let\PY@ul=\relax \let\PY@tc=\relax%
    \let\PY@bc=\relax \let\PY@ff=\relax}
\def\PY@tok#1{\csname PY@tok@#1\endcsname}
\def\PY@toks#1+{\ifx\relax#1\empty\else%
    \PY@tok{#1}\expandafter\PY@toks\fi}
\def\PY@do#1{\PY@bc{\PY@tc{\PY@ul{%
    \PY@it{\PY@bf{\PY@ff{#1}}}}}}}
\def\PY#1#2{\PY@reset\PY@toks#1+\relax+\PY@do{#2}}

\expandafter\def\csname PY@tok@w\endcsname{\def\PY@tc##1{\textcolor[rgb]{0.73,0.73,0.73}{##1}}}
\expandafter\def\csname PY@tok@c\endcsname{\let\PY@it=\textit\def\PY@tc##1{\textcolor[rgb]{0.25,0.50,0.50}{##1}}}
\expandafter\def\csname PY@tok@cp\endcsname{\def\PY@tc##1{\textcolor[rgb]{0.74,0.48,0.00}{##1}}}
\expandafter\def\csname PY@tok@k\endcsname{\let\PY@bf=\textbf\def\PY@tc##1{\textcolor[rgb]{0.00,0.50,0.00}{##1}}}
\expandafter\def\csname PY@tok@kp\endcsname{\def\PY@tc##1{\textcolor[rgb]{0.00,0.50,0.00}{##1}}}
\expandafter\def\csname PY@tok@kt\endcsname{\def\PY@tc##1{\textcolor[rgb]{0.69,0.00,0.25}{##1}}}
\expandafter\def\csname PY@tok@o\endcsname{\def\PY@tc##1{\textcolor[rgb]{0.40,0.40,0.40}{##1}}}
\expandafter\def\csname PY@tok@ow\endcsname{\let\PY@bf=\textbf\def\PY@tc##1{\textcolor[rgb]{0.67,0.13,1.00}{##1}}}
\expandafter\def\csname PY@tok@nb\endcsname{\def\PY@tc##1{\textcolor[rgb]{0.00,0.50,0.00}{##1}}}
\expandafter\def\csname PY@tok@nf\endcsname{\def\PY@tc##1{\textcolor[rgb]{0.00,0.00,1.00}{##1}}}
\expandafter\def\csname PY@tok@nc\endcsname{\let\PY@bf=\textbf\def\PY@tc##1{\textcolor[rgb]{0.00,0.00,1.00}{##1}}}
\expandafter\def\csname PY@tok@nn\endcsname{\let\PY@bf=\textbf\def\PY@tc##1{\textcolor[rgb]{0.00,0.00,1.00}{##1}}}
\expandafter\def\csname PY@tok@ne\endcsname{\let\PY@bf=\textbf\def\PY@tc##1{\textcolor[rgb]{0.82,0.25,0.23}{##1}}}
\expandafter\def\csname PY@tok@nv\endcsname{\def\PY@tc##1{\textcolor[rgb]{0.10,0.09,0.49}{##1}}}
\expandafter\def\csname PY@tok@no\endcsname{\def\PY@tc##1{\textcolor[rgb]{0.53,0.00,0.00}{##1}}}
\expandafter\def\csname PY@tok@nl\endcsname{\def\PY@tc##1{\textcolor[rgb]{0.63,0.63,0.00}{##1}}}
\expandafter\def\csname PY@tok@ni\endcsname{\let\PY@bf=\textbf\def\PY@tc##1{\textcolor[rgb]{0.60,0.60,0.60}{##1}}}
\expandafter\def\csname PY@tok@na\endcsname{\def\PY@tc##1{\textcolor[rgb]{0.49,0.56,0.16}{##1}}}
\expandafter\def\csname PY@tok@nt\endcsname{\let\PY@bf=\textbf\def\PY@tc##1{\textcolor[rgb]{0.00,0.50,0.00}{##1}}}
\expandafter\def\csname PY@tok@nd\endcsname{\def\PY@tc##1{\textcolor[rgb]{0.67,0.13,1.00}{##1}}}
\expandafter\def\csname PY@tok@s\endcsname{\def\PY@tc##1{\textcolor[rgb]{0.73,0.13,0.13}{##1}}}
\expandafter\def\csname PY@tok@sd\endcsname{\let\PY@it=\textit\def\PY@tc##1{\textcolor[rgb]{0.73,0.13,0.13}{##1}}}
\expandafter\def\csname PY@tok@si\endcsname{\let\PY@bf=\textbf\def\PY@tc##1{\textcolor[rgb]{0.73,0.40,0.53}{##1}}}
\expandafter\def\csname PY@tok@se\endcsname{\let\PY@bf=\textbf\def\PY@tc##1{\textcolor[rgb]{0.73,0.40,0.13}{##1}}}
\expandafter\def\csname PY@tok@sr\endcsname{\def\PY@tc##1{\textcolor[rgb]{0.73,0.40,0.53}{##1}}}
\expandafter\def\csname PY@tok@ss\endcsname{\def\PY@tc##1{\textcolor[rgb]{0.10,0.09,0.49}{##1}}}
\expandafter\def\csname PY@tok@sx\endcsname{\def\PY@tc##1{\textcolor[rgb]{0.00,0.50,0.00}{##1}}}
\expandafter\def\csname PY@tok@m\endcsname{\def\PY@tc##1{\textcolor[rgb]{0.40,0.40,0.40}{##1}}}
\expandafter\def\csname PY@tok@gh\endcsname{\let\PY@bf=\textbf\def\PY@tc##1{\textcolor[rgb]{0.00,0.00,0.50}{##1}}}
\expandafter\def\csname PY@tok@gu\endcsname{\let\PY@bf=\textbf\def\PY@tc##1{\textcolor[rgb]{0.50,0.00,0.50}{##1}}}
\expandafter\def\csname PY@tok@gd\endcsname{\def\PY@tc##1{\textcolor[rgb]{0.63,0.00,0.00}{##1}}}
\expandafter\def\csname PY@tok@gi\endcsname{\def\PY@tc##1{\textcolor[rgb]{0.00,0.63,0.00}{##1}}}
\expandafter\def\csname PY@tok@gr\endcsname{\def\PY@tc##1{\textcolor[rgb]{1.00,0.00,0.00}{##1}}}
\expandafter\def\csname PY@tok@ge\endcsname{\let\PY@it=\textit}
\expandafter\def\csname PY@tok@gs\endcsname{\let\PY@bf=\textbf}
\expandafter\def\csname PY@tok@gp\endcsname{\let\PY@bf=\textbf\def\PY@tc##1{\textcolor[rgb]{0.00,0.00,0.50}{##1}}}
\expandafter\def\csname PY@tok@go\endcsname{\def\PY@tc##1{\textcolor[rgb]{0.53,0.53,0.53}{##1}}}
\expandafter\def\csname PY@tok@gt\endcsname{\def\PY@tc##1{\textcolor[rgb]{0.00,0.27,0.87}{##1}}}
\expandafter\def\csname PY@tok@err\endcsname{\def\PY@bc##1{\setlength{\fboxsep}{0pt}\fcolorbox[rgb]{1.00,0.00,0.00}{1,1,1}{\strut ##1}}}
\expandafter\def\csname PY@tok@kc\endcsname{\let\PY@bf=\textbf\def\PY@tc##1{\textcolor[rgb]{0.00,0.50,0.00}{##1}}}
\expandafter\def\csname PY@tok@kd\endcsname{\let\PY@bf=\textbf\def\PY@tc##1{\textcolor[rgb]{0.00,0.50,0.00}{##1}}}
\expandafter\def\csname PY@tok@kn\endcsname{\let\PY@bf=\textbf\def\PY@tc##1{\textcolor[rgb]{0.00,0.50,0.00}{##1}}}
\expandafter\def\csname PY@tok@kr\endcsname{\let\PY@bf=\textbf\def\PY@tc##1{\textcolor[rgb]{0.00,0.50,0.00}{##1}}}
\expandafter\def\csname PY@tok@bp\endcsname{\def\PY@tc##1{\textcolor[rgb]{0.00,0.50,0.00}{##1}}}
\expandafter\def\csname PY@tok@fm\endcsname{\def\PY@tc##1{\textcolor[rgb]{0.00,0.00,1.00}{##1}}}
\expandafter\def\csname PY@tok@vc\endcsname{\def\PY@tc##1{\textcolor[rgb]{0.10,0.09,0.49}{##1}}}
\expandafter\def\csname PY@tok@vg\endcsname{\def\PY@tc##1{\textcolor[rgb]{0.10,0.09,0.49}{##1}}}
\expandafter\def\csname PY@tok@vi\endcsname{\def\PY@tc##1{\textcolor[rgb]{0.10,0.09,0.49}{##1}}}
\expandafter\def\csname PY@tok@vm\endcsname{\def\PY@tc##1{\textcolor[rgb]{0.10,0.09,0.49}{##1}}}
\expandafter\def\csname PY@tok@sa\endcsname{\def\PY@tc##1{\textcolor[rgb]{0.73,0.13,0.13}{##1}}}
\expandafter\def\csname PY@tok@sb\endcsname{\def\PY@tc##1{\textcolor[rgb]{0.73,0.13,0.13}{##1}}}
\expandafter\def\csname PY@tok@sc\endcsname{\def\PY@tc##1{\textcolor[rgb]{0.73,0.13,0.13}{##1}}}
\expandafter\def\csname PY@tok@dl\endcsname{\def\PY@tc##1{\textcolor[rgb]{0.73,0.13,0.13}{##1}}}
\expandafter\def\csname PY@tok@s2\endcsname{\def\PY@tc##1{\textcolor[rgb]{0.73,0.13,0.13}{##1}}}
\expandafter\def\csname PY@tok@sh\endcsname{\def\PY@tc##1{\textcolor[rgb]{0.73,0.13,0.13}{##1}}}
\expandafter\def\csname PY@tok@s1\endcsname{\def\PY@tc##1{\textcolor[rgb]{0.73,0.13,0.13}{##1}}}
\expandafter\def\csname PY@tok@mb\endcsname{\def\PY@tc##1{\textcolor[rgb]{0.40,0.40,0.40}{##1}}}
\expandafter\def\csname PY@tok@mf\endcsname{\def\PY@tc##1{\textcolor[rgb]{0.40,0.40,0.40}{##1}}}
\expandafter\def\csname PY@tok@mh\endcsname{\def\PY@tc##1{\textcolor[rgb]{0.40,0.40,0.40}{##1}}}
\expandafter\def\csname PY@tok@mi\endcsname{\def\PY@tc##1{\textcolor[rgb]{0.40,0.40,0.40}{##1}}}
\expandafter\def\csname PY@tok@il\endcsname{\def\PY@tc##1{\textcolor[rgb]{0.40,0.40,0.40}{##1}}}
\expandafter\def\csname PY@tok@mo\endcsname{\def\PY@tc##1{\textcolor[rgb]{0.40,0.40,0.40}{##1}}}
\expandafter\def\csname PY@tok@ch\endcsname{\let\PY@it=\textit\def\PY@tc##1{\textcolor[rgb]{0.25,0.50,0.50}{##1}}}
\expandafter\def\csname PY@tok@cm\endcsname{\let\PY@it=\textit\def\PY@tc##1{\textcolor[rgb]{0.25,0.50,0.50}{##1}}}
\expandafter\def\csname PY@tok@cpf\endcsname{\let\PY@it=\textit\def\PY@tc##1{\textcolor[rgb]{0.25,0.50,0.50}{##1}}}
\expandafter\def\csname PY@tok@c1\endcsname{\let\PY@it=\textit\def\PY@tc##1{\textcolor[rgb]{0.25,0.50,0.50}{##1}}}
\expandafter\def\csname PY@tok@cs\endcsname{\let\PY@it=\textit\def\PY@tc##1{\textcolor[rgb]{0.25,0.50,0.50}{##1}}}

\def\PYZbs{\char`\\}
\def\PYZus{\char`\_}
\def\PYZob{\char`\{}
\def\PYZcb{\char`\}}
\def\PYZca{\char`\^}
\def\PYZam{\char`\&}
\def\PYZlt{\char`\<}
\def\PYZgt{\char`\>}
\def\PYZsh{\char`\#}
\def\PYZpc{\char`\%}
\def\PYZdl{\char`\$}
\def\PYZhy{\char`\-}
\def\PYZsq{\char`\'}
\def\PYZdq{\char`\"}
\def\PYZti{\char`\~}
% for compatibility with earlier versions
\def\PYZat{@}
\def\PYZlb{[}
\def\PYZrb{]}
\makeatother


    % Exact colors from NB
    \definecolor{incolor}{rgb}{0.0, 0.0, 0.5}
    \definecolor{outcolor}{rgb}{0.545, 0.0, 0.0}



    
    % Prevent overflowing lines due to hard-to-break entities
    \sloppy 
    % Setup hyperref package
    \hypersetup{
      breaklinks=true,  % so long urls are correctly broken across lines
      colorlinks=true,
      urlcolor=urlcolor,
      linkcolor=linkcolor,
      citecolor=citecolor,
      }
    % Slightly bigger margins than the latex defaults
    
    \geometry{verbose,tmargin=1in,bmargin=1in,lmargin=1in,rmargin=1in}
    
    

    \begin{document}
    
    
    \maketitle
    
    

    
    \hypertarget{module-4-final-project}{%
\section{Module 4 Final Project}\label{module-4-final-project}}

\hypertarget{introduction}{%
\subsection{Introduction}\label{introduction}}

In this lesson, we'll review all of the guidelines and specifications
for the final project for Module 4.

\hypertarget{objectives}{%
\subsection{Objectives}\label{objectives}}

You will be able to: * Describe all required aspects of the final
project for Module 4 * Describe all required deliverables * Describe
what constitutes a successful project * Describe what the experience of
the project review should be like

    \hypertarget{final-project-summary}{%
\subsection{Final Project Summary}\label{final-project-summary}}

You've made it all the way through one of the toughest modules of this
course, and demonstrated a solid understanding of the principles of Deep
Learning. You must have an amazing brain in your head!

For this module's final project, you'll put everything you've learned
together to build a Deep Neural Network that trains on a large dataset
for classification on a non-trivial task! This project will include:

\begin{itemize}
\tightlist
\item
  Selecting a problem
\item
  Sourcing an appropriate dataset
\item
  Setting up your project (directory structure, etc)
\item
  Building, training, tuning, and evaluating a Deep Neural Network
\item
  Explaining your methodology and findings in a clear, concise manner
\end{itemize}

Let's get started by examining the dataset requirements for this
project.

\hypertarget{the-dataset}{%
\subsection{The Dataset}\label{the-dataset}}

For this module's project, the dataset will be heavily tied to the
problem you are trying to solve. We recommend that you base your project
around one of the three following subdomains in Deep Learning which you
how have experience with:

\begin{itemize}
\tightlist
\item
  Traditional analytics (classification or regression tasks)
\item
  Computer Vision
\item
  Text Classification/NLP
\end{itemize}

\hypertarget{picking-a-reasonable-problem}{%
\subsubsection{Picking a Reasonable
Problem}\label{picking-a-reasonable-problem}}

Note that in respect to this project, all datasets and problems are not
created equal--while you could likely build a working model for just
about any dataset you find in theory, in practice, you'll find that many
datasets have dimensionality issues that make them intractable for
training without spending hundreds or even thousands of dollars training
your model on a professional server cluster filled with high-end GPUs.

A good litmus test for checking a project's feasibility is to head over
to Kaggle or do a quick Google search to see if anyone else has already
solved this problem. If they have, then it's likely that you can, too!
Just remember, you only have access to a local machine for this project,
not a server cluster, so the problem should be one that can be solved on
your own laptop!

Here are some caveats you should consider when selecting your dataset:

\hypertarget{a-note-on-computer-vision-datasets}{%
\paragraph{A Note on Computer Vision
Datasets}\label{a-note-on-computer-vision-datasets}}

\textbf{Try to stay away from color images, or images that are larger
than 40x40 pixels}. Convolutional Layers are very expensive, and most
models can still make successful classifications on grainy,
black-and-white images just fine. Pictures that are too large add a
bunch of needless dimensionality to the model--remember, every single
pixel in the model is a dimension! Similarly, since color images are
Rank-3 Tensors (3-dimensional arrays contain Red, Green, and Blue values
for each pixel), they also needlessly triple dimensionality without
adding important information to your model in most cases.

\hypertarget{aim-for-a-proof-of-concept}{%
\paragraph{Aim for a Proof of
Concept}\label{aim-for-a-proof-of-concept}}

With Deep Learning, data is king--the more of it, the better. However,
the goal of this project isn't to build the best model possible--it's to
demonstrate your understanding by building a model that works. The true
goal of this project is to gain experience with Deep Learning and to
build a portfolio project you can be proud of, and that doesn't
necessarily require a model with incredibly high accuracy. You should
try to avoid datasets and model architectures that won't run in
reasonable time on your own machine. For many problems, this means
downsampling your dataset and only training on a portion of it. Once
you're absolutely sure that you've found the best possible architecture
and other hyperparameters for your model, then consider training your
model on your entire dataset overnight (or, as larger portion of the
dataset that will still run in a feasible amount of time).

At the end of the day, we want to see your thought process as you
iterate and improve on a model. A Project that achieves a lower level of
accuracy but has clearly iterated on the model and the problem until it
found the best possible approach is more impressive than a model with
high accuracy that did not iteration. We're not just interested in
seeing you finish a model--we want to see that you understand them, and
can use this knowledge to try and make them better and better!

\hypertarget{preexisting-datasets}{%
\paragraph{Preexisting Datasets}\label{preexisting-datasets}}

As you start exploring datasets that are appropriate for Deep Learning,
you'll probably start to see some of the same datasets mentioned again
and again, such as CIFAR10. For this project, it is acceptable to use
popular preexisting datasets. \textbf{It is also acceptable to use
datasets that you've found on popular websites such as Kaggle--you'll
find a very active Deep Learning community on that website, and plenty
of awesome datasets that are perfect for this sort of project!}

\hypertarget{sourcing-your-own-dataset}{%
\paragraph{Sourcing Your Own Dataset}\label{sourcing-your-own-dataset}}

If you so choose, you are also welcome to source your own dataset for
this project, although we strongly advise you to think carefully about
whether this is worth the time before attempting this! You'll likely
need thousands of examples, and scraping google images or other websites
can sometimes be more trouble than it's worth. If you feel up to the
task, then you are more than welcome to source your own dataset through
scraping. However, we strongly encourage you to search the web for
preexisting datasets that would work for your chosen task before
attempting to source your own, as they likely already exist, and will
save you a ton of time debugging your scraping code or getting an API to
work. \textbf{If you plan on sourcing your own dataset for this project,
please clear this with your instuctor first!}

\#\#\#\# Avoid Generative Models

After the end of the Deep Learning module, you may be tempted to try
building a Generative Model such as a Generative Adversarial Network,
Variation Autoencoder, or Sequence Generation model. Although you
theoretically know enough to attempt such problems, in practice, these
models are much too computationally intensive for you to see any
meaningful results on a local machine in the time allotted. For
reference, most GANs for image generation need to train for a minimum of
3 days straight on a server cluster with 64 high-end GPUs before showing
any meaningful results! The other issue with generative models is that
they are unsupervised, so it is impossible to generate any sort of
accuracy or performance metrics. \textbf{For this reason, you must stick
to supervised learning and only build discriminative models for this
project. No generative models will be approved.}

\hypertarget{the-deliverables}{%
\subsection{The Deliverables}\label{the-deliverables}}

    \hypertarget{the-process}{%
\subsection{The Process}\label{the-process}}

\hypertarget{getting-started}{%
\subsubsection{1. Getting Started}\label{getting-started}}

Please start by reviewing this document. If you have any questions,
please ask them in slack ASAP so (a) we can answer the questions and (b)
so we can update this repository to make it clearer.

Once you're done with the rest of the module, please start on the
project. Do that by forking this repository, cloning it locally, and
working in the student.ipynb file. Make sure to also add and commit a
pdf of your presentation to the repository with a file name of
\texttt{presentation.pdf}.

\hypertarget{the-project-review}{%
\subsubsection{2. The Project Review}\label{the-project-review}}

\begin{quote}
\textbf{When you start on the project, please also reach out to an
instructor immediately to schedule your project review} (if you're not
sure who to schedule with, please ask in slack!)
\end{quote}

\hypertarget{what-to-expect-from-the-project-review}{%
\paragraph{What to expect from the Project
Review}\label{what-to-expect-from-the-project-review}}

Project reviews are focused on preparing you for technical interviews.
Treat project reviews as if they were technical interviews, in both
attitude and technical presentation \emph{(sometimes technical
interviews will feel arbitrary or unfair - if you want to get the job,
commentiing on that is seldom a good choice)}.

The project review is comprised of a 45 minute 1:1 session with one of
the instructors. During your project review, be prepared to:

\hypertarget{deliver-your-pdf-presentation-to-a-non-technical-stakeholder.}{%
\paragraph{1. Deliver your PDF presentation to a non-technical
stakeholder.}\label{deliver-your-pdf-presentation-to-a-non-technical-stakeholder.}}

In this phase of the review (\textasciitilde{}10 mins) your instructor
will play the part of a non-technical stakeholder that you are
presenting your findings to. The presentation should not exceed 5
minutes, giving the ``stakeholder'' 5 minutes to ask questions.

In the first half of the presentation (2-3 mins), you should summarize
your methodology in a way that will be comprehensible to someone with no
background in data science and that will increase their confidence in
you and your findings. In the second half (the remaining 2-3 mins) you
should summarize your findings and be ready to answer a couple of
non-technical questions from the audience. The questions might relate to
technical topics (sampling bias, confidence, etc) but will be asked in a
non-technical way and need to be answered in a way that does not assume
a background in statistics or machine learning. You can assume a smart,
business stakeholder, with a non-quantitative college degree.

\hypertarget{go-through-the-jupyter-notebook-answering-questions-about-how-you-made-certain-decisions.-be-ready-to-explain-things-like}{%
\paragraph{2. Go through the Jupyter Notebook, answering questions about
how you made certain decisions. Be ready to explain things
like:}\label{go-through-the-jupyter-notebook-answering-questions-about-how-you-made-certain-decisions.-be-ready-to-explain-things-like}}

\begin{verbatim}
* "how did you pick the question(s) that you did?"
* "why are these questions important from a business perspective?"
* "how did you decide on the data cleaning options you performed?"
* "why did you choose a given method or library?"
* "why did you select those visualizations and what did you learn from each of them?"
* "why did you pick those features as predictors?"
* "how would you interpret the results?"
* "how confident are you in the predictive quality of the results?"
* "what are some of the things that could cause the results to be wrong?"
\end{verbatim}

Think of the second phase of the review (\textasciitilde{}30 mins) as a
technical boss reviewing your work and asking questions about it before
green-lighting you to present to the business team. You should practice
using the appropriate technical vocabulary to explain yourself. Don't be
surprised if the instructor jumps around or sometimes cuts you off -
there is a lot of ground to cover, so that may happen.

If any requirements are missing or if significant gaps in understanding
are uncovered, be prepared to do one or all of the following: * Perform
additional data cleanup, visualization, feature selection, modeling
and/or model validation * Submit an improved version * Meet again for
another Project Review

What won't happen: * You won't be yelled at, belittled, or scolded * You
won't be put on the spot without support * There's nothing you can do to
instantly fail or blow it

\textbf{Please note: We need to receive the URL of your repository at
least 24 hours before your review so we can look at your materials in
advance.}

    \hypertarget{requirements}{%
\subsection{Requirements}\label{requirements}}

This section outlines the rubric we'll use to evaluate your project.

\hypertarget{technical-report-must-haves}{%
\subsubsection{1. Technical Report
Must-Haves}\label{technical-report-must-haves}}

Your jupyter notebook should include all code written for this project.
This includes any code for sourcing, cleaning, and preprocessing data.
Your technical report should also contain a record of the various
different hyperparameters you tried during the tuning process, and the
results each achieved. Any data scientist given your technical report
should be able to reproduce every step you took during the project from
start to finish and achieve the same results, so don't forget to set a
random seed for reproducibility!

As always, your jupyter notebook should be well-organized and easy to
read, with clean, well-commented code as necessary.

\hypertarget{non-technical-presentation-must-haves}{%
\subsubsection{2. Non-Technical Presentation
Must-Haves}\label{non-technical-presentation-must-haves}}

Just as with the other projects, you should also complete a 5-10 slide
PowerPoint or Google Slides presentation that explains your problem,
methodology, and results to non-technical stakeholders. This can be
especially hard with Deep Learning--try not to get bogged down with
technical jargon! Your slide deck should take \textasciitilde{}5 minutes
to go through, and should contain graphics and avoid long blocks of text
or code when possible.

\textbf{\emph{HINT}}: Keras provides
\href{https://keras.io/visualization/}{excellent documentation} on how
to create a visualization of your neural network's architecture!

\hypertarget{blog-post}{%
\subsubsection{3. Blog Post}\label{blog-post}}

Please also write a blog post about your experience working on this
project. This blog post should provide insight into the problem you are
trying to solve and your dataset, any preprocessing steps required, and
your approach to building and iteratively tuning your model. It should
also contain an explanation of any problems, obstacles, or surprises you
encountered during this project. The blog post should be between
800-1500 words and should be targeted at your peers - aspiring data
scientists.

    \hypertarget{summary}{%
\subsection{Summary}\label{summary}}

The end of module projects and project reviews are a critical part of
the program. They give you a chance to both bring together all the
skills you've learned into realistic projects and to practice key
``business judgement'' and communication skills that you otherwise might
not get as much practice with.

The projects are serious and important. They are not graded, but they
can be passed and they can be failed. Take the project seriously, put
the time in, ask for help from your peers or instructors early and often
if you need it, and treat the review as a job interview and you'll do
great. We're rooting for you to succeed and we're only going to ask you
to take a review again if we believe that you need to. We'll also
provide open and honest feedback so you can improve as quickly and
efficiently as possible.

We don't expect you to remember all of the terms or to get all of the
answers right. If in doubt, be honest. If you don't know something, say
so. If you can't remember it, just say so. It's very unusual for someone
to complete a project review without being asked a question they're
unsure of, we know you might be nervous which may affect your
performance. Just be as honest, precise and focused as you can be, and
you'll do great!


    % Add a bibliography block to the postdoc
    
    
    
    \end{document}
